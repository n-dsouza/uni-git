\documentclass[12pt]{article}

\usepackage[utf8]{inputenc} % Required for inputting international characters
\usepackage[T1]{fontenc} % Output font encoding for international characters
\usepackage{mathpazo} % Palatino font


%%% PAGE DIMENSIONS
\usepackage[margin=2.54cm]{geometry} % 1.27 is narrower!
\geometry{a4paper} 

%%% Font set to times new roman
\usepackage{times}
%%% Really good graphics package
\usepackage{graphicx}

% Insert {PDFS}
\usepackage{pdfpages}

% Change section spacing, remove the spacing
\usepackage[compact]{titlesec}
\titlespacing*{\section}{0pt}{*0}{*0}
\titlespacing*{\subsection}{0pt}{*0}{*0}
\titlespacing*{\subsubsection}{0pt}{*0}{*0}

%%% Better floats
\usepackage{float}


\usepackage{tikz}
\usepackage{wrapfig}

%%% PACKAGES
\usepackage{booktabs} % for much better looking tables
\usepackage{amsmath} % for better maths
\usepackage{paralist} % very flexible & customisable lists (eg. enumerate/itemize, etc.)
\usepackage{verbatim} % adds environment for commenting out blocks of text & for better verbatim
\usepackage{subfig} % make it possible to include more than one captioned figure/table in a single float
\usepackage{mathtools} % matrices
\usepackage{blkarray, bigstrut} % matrices

\usepackage[framed,numbered]{matlab-prettifier} % enable inserting matlab code.
\usepackage[parfill]{parskip}
%\addtolength{\jot}{1em}
\usepackage{amssymb}
\usepackage{cancel}
\usepackage{color}
\usepackage{listings}
\usepackage{pdflscape} % Enabling rotating pages into a landscape orientation.


%%% HEADERS & FOOTERS
\usepackage{fancyhdr} % This should be set AFTER setting up the page geometry
\pagestyle{fancy} % options: empty , plain , fancy
\renewcommand{\headrulewidth}{0pt} % customise the layout...
\lhead{}\chead{}\rhead{}
\lfoot{}\cfoot{\thepage}\rfoot{}

%%% SECTION TITLE APPEARANCE
\usepackage{sectsty}
\renewcommand{\thesubsection}{\thesection.\alph{subsection}}

%%% ToC (table of contents) APPEARANCE
\usepackage[nottoc,notlof,notlot]{tocbibind} % Put the bibliography in the ToC
\usepackage[titles,subfigure]{tocloft} % Alter the style of the Table of Contents
\renewcommand{\cftsecfont}{\rmfamily\mdseries\upshape}
\renewcommand{\cftsecpagefont}{\rmfamily\mdseries\upshape} % No bold!

%%% Hyperlinking 
\usepackage{hyperref}
\begin{document}

\begin{titlepage} % Suppresses displaying the page number on the title page and the subsequent page counts as page 1
	\newcommand{\HRule}{\rule{\linewidth}{0.5mm}} % Defines a new command for horizontal lines, change thickness here
	\center % Centre everything on the page
	
	%------------------------------------------------
	%	Headings
    %------------------------------------------------
    
	\textsc{\LARGE 2019}\\[1.5cm] % Main heading such as the name of your university/college
	\textsc{\Large Semester 1}\\[0.5cm] % Major heading such as course name
	% \textsc{\large Department of Engineering Science}\\[0.5cm] % Minor heading such as course title
	
	%------------------------------------------------
	%	Title
	%------------------------------------------------
	
	\HRule\\[0.4cm]
	{\huge\bfseries ENGSCI 760 --- Assignment 1}\\[0.4cm] % Title of your document
	\HRule\\[1cm]
	
	%------------------------------------------------
	%	Author(s)
	%------------------------------------------------
	
    \begin{minipage}[t]{0.4\textwidth}
        \vspace{0pt}
		\begin{center}
			\large
			Noel \textsc{D'Souza}\\ % Your name
			\textit{ndso092}\\
			\textit{449609993}
		\end{center}
	\end{minipage}
	~
    % \begin{minipage}[t]{0.4\textwidth}
    %     \vspace{0pt}\raggedright
	% 	\begin{flushright}
	% 		\large
	% 		\textit{Employer Details}\\
    %         PwC \textsc{New Zealand} % Supervisor's name
    %         \linebreak \linebreak
    %         PwC Tower\\
    %         Level 22, 188 Quay St\\
    %         Auckland, 1010
	% 	\end{flushright}
	% \end{minipage}

	% If you don't want a supervisor, uncomment the two lines below and comment the code above
	%{\large\textit{Author}}\\
	%John \textsc{Smith} % Your name
	
	%------------------------------------------------
	%	Date
	%------------------------------------------------
	
	\vfill\vfill\vfill % Position the date 3/4 down the remaining page
	
    {\large\today} % Date, change the \today to a set date if you want to be precise
    % {\large 7 August, 2019}
	
	%------------------------------------------------
	%	Logo
	%------------------------------------------------
	
    % \vfill
    
    % \begin{minipage}[t]{0.4\textwidth}
    %     \vspace{0pt}
    %     \begin{flushleft}
    %     \includegraphics[width=0.9\textwidth]{engtitle.png}\\[1cm]
	% 	\end{flushleft}
	% \end{minipage}
	% ~
    % \begin{minipage}[t]{0.4\textwidth}
    %     \vspace{0pt}\raggedright
	% 	\begin{flushright}
    %         \includegraphics[width=0.45\textwidth]{pwc.png}\\[1cm]        
	% 	\end{flushright}
	% \end{minipage}


    % \includegraphics[width=0.4\textwidth]{engtitle.png}\\[1cm] % Include a department/university logo - this will require the graphicx package
	 
	%----------------------------------------------------------------------------------------
	
	\vfill % Push the date up 1/4 of the remaining page
	
\end{titlepage}

%------------------------------------------------
%   Summary
%------------------------------------------------

\section{Joint Distributions}

\subsection{Pairwise Independence?}

Events A, B and C \textbf{are} pairwise independent. This is because Event A is independent of B and is also independent of C. Events B and C are also independent of each other. This means the outcomes in any \textbf{pair of events} are independent of each other

\subsection{Mutual indepentence?}

Events A, B and C \textbf{are not} mutually independent. This is because if we know the outcome of Event B and Event C, then we know with 100\% certainty the outcome of Event A. In other words, Event A is \textbf{dependent} on the intersection of Events B and C.

\section{Markov Chains}

\subsection{One-Step Transition Matrix}
\begin{equation*}
	\mathbf{P}=
	\begin{blockarray}{*{5}{c} l}
	  \begin{block}{*{5}{>{$\footnotesize}c<{$}} l}
		U & G & A & P & S & \\
	  \end{block}
	  \begin{block}{[*{5}{c}]>{$\footnotesize}l<{$}}
		0 & 0.4 & 0.3 & 0.2 & 0.1 \bigstrut[t]& U \\
		0 & 1 & 0 & 0 & 0 & G \\
		0 & 0 & 1 & 0 & 0 & A \\
		0 & 0.4 & 0.3 & 0.2 & 0.1 & P \\
		0 & 0 & 0 & 0 & 1 & S \\
	  \end{block}
	\end{blockarray}
\end{equation*}

\subsection{Limiting Distribution}
	
\begin{align*}
	\text{Proportion of good items} & = \text{Probability that a given item is good} \\
	& = \text{Pr(item reaches G in step 1 \textit{or} item reaches G in step 2 \textit{or} ...)} \\
	& = \text{Pr(item good in step 1) + Pr(item good in step 2) + ...}\\
	& = 0.4 + (0.2)(0.4) + (0.2)(0.2)(0.4) + (0.2)(0.2)^2(0.4) + ... \\
	& = 0.4 + (0.2)(0.4)\sum_{k=0}^{\infty}{(0.2)^k}\\
	& = 0.4 + \frac{0.08}{0.8}\\
	\text{Pr}(X_t = \text{G})& = 0.5
\end{align*}

\begin{align*}
	\text{Proportion of average items} & = \text{Pr(average in step 1) + Pr(average in step 2) + ...}\\
	& = 0.3 + (0.2)(0.3) + (0.2)(0.2)(0.3) + (0.2)(0.2)^2(0.3) + ... \\
	& = 0.3 + (0.2)(0.3)\sum_{k=0}^{\infty}{(0.2)^k}\\
	& = 0.3 + \frac{0.06}{0.8}\\
	\text{Pr}(X_t = \text{A})& = 0.375
\end{align*}

\begin{align*}
	\text{Proportion of scrapped items} & = \text{Pr(scrap in step 1) + Pr(scrap in step 2) + ...}\\
	& = 0.1 + (0.2)(0.1) + (0.2)(0.2)(0.1) + (0.2)(0.2)^2(0.1) + ... \\
	& = 0.1 + (0.2)(0.1)\sum_{k=0}^{\infty}{(0.2)^k}\\
	& = 0.1 + \frac{0.02}{0.8}\\
	\text{Pr}(X_t = \text{S})& = 0.125
\end{align*}

\subsection{Expected Profit From Unfinished Item}

\begin{align*}
\text{Note that:}\\
m_{ij} & = \sum_{k=1}^{n} p_{ik}(m_{kj}+1) \text{,\qquad \qquad} i \neq j\\
& = 1 + \sum_{k=1}^{n} p_{ik}m_{kj} \text{,\qquad \qquad} i \neq j\\
& \\
m_{\text{UG}} &= 1 + p_{\text{UU}}m_{\text{UG}} + p_{\text{UG}}m_{\text{GG}} + p_{\text{UA}}m_{\text{AG}} + p_{\text{UP}}m_{\text{PG}} + p_{\text{US}}m_{\text{SG}}\\
&= 1 + (0)m_{\text{UG}} + (0.4)m_{\text{GG}} + (0.3)m_{\text{AG}} + (0.2)m_{\text{PG}} + (0.1)m_{\text{SG}}\\
& = 1 + 0 + (0.4)(0) + (0.3)(0) + 0.2m_{\text{PG}} + (0.1)(0)\\
& = 1 + 0.2m_{\text{PG}}\\
& \\
\text{Similarly:}\\
m_{\text{UA}} &= 1 + 0.2m_{\text{PG}}\\
m_{\text{US}} &= 1 + 0.2m_{\text{PG}}\\
& \\
\text{Therefore:}\\
m_{\text{UG}} &= m_{\text{UA}} = m_{\text{US}} = 1.25\\
& \\
\text{Hence:}\\
\text{Expected profit} & = (50\times0.5) + (40\times0.375) - (10\times1.25) - 20\\
& = 25 + 15 - 12.5 - 20\\
& = \$7.5
\end{align*}

\subsection{Updated One-Step Transition Matrix}
\begin{equation*}
	\mathbf{P_{new}}=
	\begin{blockarray}{*{7}{c} l}
	  \begin{block}{*{7}{>{$\footnotesize}c<{$}} l}
		U & G & A & S & $P_1$ & $P_2$ & $P_3$ & \\
	  \end{block}
	  \begin{block}{[*{7}{c}]>{$\footnotesize}l<{$}}
		0 & 0.4 & 0.3 & 0.1& 0.2 & 0 & 0\bigstrut[t]& U \\
		0 & 1 & 0 & 0 & 0 & 0 & 0 & G \\
		0 & 0 & 1 & 0 & 0 & 0 & 0 & A \\
		0 & 0 & 0 & 1 & 0 & 0 & 0 & S \\
		0 & 0.4 & 0.3 & 0.1 & 0 & 0.2 & 0 & $P_1$ \\
		0 & 0.4 & 0.3 & 0.1 & 0 & 0 & 0.2 & $P_2$ \\
		0 & 0 & 0 & 0 & 0 & 0 & 1 & $P_3$ \\
	  \end{block}
	\end{blockarray}
\end{equation*}

\section{Hidden Markov Model}

\subsection{createTransitions}
\subsection{createEmissions}
\subsection{HMM}
\subsection{main}

\end{document}
