\documentclass[12pt]{article}

\usepackage[utf8]{inputenc} % Required for inputting international characters
\usepackage[T1]{fontenc} % Output font encoding for international characters
\usepackage{mathpazo} % Palatino font


%%% PAGE DIMENSIONS
\usepackage[margin=2.54cm]{geometry} % 1.27 is narrower!
\geometry{a4paper} 

%%% Font set to times new roman
\usepackage{times}
%%% Really good graphics package
\usepackage{graphicx}

% Insert {PDFS}
\usepackage{pdfpages}

% Change section spacing, remove the spacing
\usepackage[compact]{titlesec}
\titlespacing*{\section}{0pt}{*0}{*0}
\titlespacing*{\subsection}{0pt}{*0}{*0}
\titlespacing*{\subsubsection}{0pt}{*0}{*0}

%%% Better floats
\usepackage{float}


\usepackage{tikz}
\usepackage{wrapfig}

%%% PACKAGES
\usepackage{booktabs} % for much better looking tables
\usepackage{amsmath} % for better maths
\usepackage{paralist} % very flexible & customisable lists (eg. enumerate/itemize, etc.)
\usepackage{verbatim} % adds environment for commenting out blocks of text & for better verbatim
\usepackage{subfig} % make it possible to include more than one captioned figure/table in a single float
\usepackage{mathtools} % matrices
\usepackage{blkarray, bigstrut} % matrices

\usepackage[framed,numbered]{matlab-prettifier} % enable inserting matlab code.
\usepackage[parfill]{parskip}
%\addtolength{\jot}{1em}
\usepackage{amssymb}
\usepackage{cancel}
\usepackage{color}
\usepackage{listings}
\usepackage{pdflscape} % Enabling rotating pages into a landscape orientation.


%%% HEADERS & FOOTERS
\usepackage{fancyhdr} % This should be set AFTER setting up the page geometry
\pagestyle{fancy} % options: empty , plain , fancy
\renewcommand{\headrulewidth}{0pt} % customise the layout...
\lhead{}\chead{}\rhead{}
\lfoot{}\cfoot{\thepage}\rfoot{}

%%% SECTION TITLE APPEARANCE
\usepackage{sectsty}
\renewcommand{\thesubsection}{\thesection.\alph{subsection}}

%%% ToC (table of contents) APPEARANCE
\usepackage[nottoc,notlof,notlot]{tocbibind} % Put the bibliography in the ToC
\usepackage[titles,subfigure]{tocloft} % Alter the style of the Table of Contents
\renewcommand{\cftsecfont}{\rmfamily\mdseries\upshape}
\renewcommand{\cftsecpagefont}{\rmfamily\mdseries\upshape} % No bold!

%%% Hyperlinking 
\usepackage{hyperref}
\begin{document}

\begin{titlepage} % Suppresses displaying the page number on the title page and the subsequent page counts as page 1
	\newcommand{\HRule}{\rule{\linewidth}{0.5mm}} % Defines a new command for horizontal lines, change thickness here
	\center % Centre everything on the page
	
	%------------------------------------------------
	%	Headings
    %------------------------------------------------
    
	\textsc{\LARGE 2019}\\[1.5cm] % Main heading such as the name of your university/college
	\textsc{\Large Semester 1}\\[0.5cm] % Major heading such as course name
	% \textsc{\large Department of Engineering Science}\\[0.5cm] % Minor heading such as course title
	
	%------------------------------------------------
	%	Title
	%------------------------------------------------
	
	\HRule\\[0.4cm]
	{\huge\bfseries FINANCE 261 --- Assignment 1}\\[0.4cm] % Title of your document
	\HRule\\[1cm]
	
	%------------------------------------------------
	%	Author(s)
	%------------------------------------------------
	
    \begin{minipage}[t]{0.4\textwidth}
        \vspace{0pt}
		\begin{center}
			\large
			Noel \textsc{D'Souza}\\ % Your name
			\textit{ndso092}\\
			\textit{449609993}
		\end{center}
	\end{minipage}
	~
    % \begin{minipage}[t]{0.4\textwidth}
    %     \vspace{0pt}\raggedright
	% 	\begin{flushright}
	% 		\large
	% 		\textit{Employer Details}\\
    %         PwC \textsc{New Zealand} % Supervisor's name
    %         \linebreak \linebreak
    %         PwC Tower\\
    %         Level 22, 188 Quay St\\
    %         Auckland, 1010
	% 	\end{flushright}
	% \end{minipage}

	% If you don't want a supervisor, uncomment the two lines below and comment the code above
	%{\large\textit{Author}}\\
	%John \textsc{Smith} % Your name
	
	%------------------------------------------------
	%	Date
	%------------------------------------------------
	
	\vfill\vfill\vfill % Position the date 3/4 down the remaining page
	
    {\large\today} % Date, change the \today to a set date if you want to be precise
    % {\large 7 August, 2019}
	
	%------------------------------------------------
	%	Logo
	%------------------------------------------------
	
    % \vfill
    
    % \begin{minipage}[t]{0.4\textwidth}
    %     \vspace{0pt}
    %     \begin{flushleft}
    %     \includegraphics[width=0.9\textwidth]{engtitle.png}\\[1cm]
	% 	\end{flushleft}
	% \end{minipage}
	% ~
    % \begin{minipage}[t]{0.4\textwidth}
    %     \vspace{0pt}\raggedright
	% 	\begin{flushright}
    %         \includegraphics[width=0.45\textwidth]{pwc.png}\\[1cm]        
	% 	\end{flushright}
	% \end{minipage}


    % \includegraphics[width=0.4\textwidth]{engtitle.png}\\[1cm] % Include a department/university logo - this will require the graphicx package
	 
	%----------------------------------------------------------------------------------------
	
	\vfill % Push the date up 1/4 of the remaining page
	
\end{titlepage}

%------------------------------------------------
%   Summary
%------------------------------------------------

\section{}
\subsection{}
A 1\% increase in the price of \textbf{Stock C} will have the greatest impact on the value of the index. This is because a Price Weighted Index is comprised of one share of each stock. Hence a 1\% increase in the price of the stock with the \textbf{largest share price} (i.e. Stock C) will contribute the most towards the index value.

\subsection{Divisor at t=0}
\begin{align*}
	\frac{P_0(A) + P_0(B) + P_0(C)}{D} &= I_0\\
	\frac{40+50+100}{D} &= 100\\
	D &= \frac{190}{100} \\
	D &= 1.9\\
\end{align*}

\subsection{Price weighted index value at t=1, and monthly \newline percentage return}
\begin{align*}
	\frac{P_1(A) + P_1(B) + P_1(C)}{D} &= I_1\\
	\frac{44+54+100}{1.9} &= I_1\\
	I_1 &= \frac{208}{1.9} \\
	I_1 &= 109.47 \text{ index value after 1 month}\\
	&\\
	\text{Monthly Return} &= \frac{109.47-100}{100}\\
	&= 9.47\%
\end{align*}

\subsection{Divisior after 2-for-1 stock split for stock C}
\begin{align*}
	\frac{P_1(A) + P_1(B) + P_1^{\text{(\textit{split})}}(C)}{D^{(\textit{split})}} &= I_1\\
	\frac{44+54+55}{D^{(\textit{split})}} &= 109.47\\
	D^{(\textit{split})} &= \frac{153}{109.47} \\
	D^{(\textit{split})} &= 1.40\\
\end{align*}

\section{}
\subsection{Initial investment and borrowing}
\begin{align*}
	\text{Position Value} &= 1000 \text{ shares} \times \$25 \text{ per share}\\
	&= \$25,000\\
	&\\
	\text{Initial Investor Equity} &= \text{Position Value} \times \text{IMR}\\
	&= \$25,000 \times 0.5\\
	&= \$12,500\\
	&\\
	\text{Amount Borrowed} &= \text{Position Value} \times (1-\text{IMR})\\
	&= \$25,000 \times (1-0.5)\\
	&= \$12,500\\
\end{align*}

\subsection{Margin call trigger price}
\begin{align*}
	\text{Maintenance Margin} &=  \frac{\text{Total Position Value} - \text{Borrowing}}{\text{Total Position Value}} \\
	0.4 &= \frac{1,000p - 12,500}{1,000p}\\
	400p &= 1,000p-12,500\\
	p &= \frac{12,500}{600}\\
	p &= \$20.83 \text{ per share is the margin call trigger price} 
\end{align*}

\subsection{Pledging T-bills}
\begin{align*}
	\text{Maintenance Margin} &=  \frac{\text{Total Value} - \text{Borrowing} + \text{Additional Contribution of T-bills}}{\text{Total Value}+ \text{Additional Contribution of T-bills}} \\
	0.4 &= \frac{(1,000\times20) - 12,500 + x}{(1,000\times20)+x} \\
	8,000 + 0.4x &= 7,500 + x\\
	0.6x &= 500\\
	x &= \$833.33 \text{ worth of T-bills must be added}\\
\end{align*}

\subsection{Extent of partial liquidation}
\begin{align*}
	\text{Maintenance Margin} &=  \frac{\text{Total Value} - \text{Borrowing}}{\text{Total Value} - \text{Value of Liquidated Shares}} \\
	0.4 &= \frac{(1,000\times20) - 12,500}{(1,000\times20)-x} \\
	8,000 - 0.4x &= 7,500\\
	0.4x &= 500\\
	x &= \$1,250 \text{ is the value of shares sold off}\\
\end{align*}

\begin{flushleft}
	\text{Given the share price of \$20 per share:}
\end{flushleft}

\begin{align*}
	\text{Number of shares sold off by broker} &= \frac{\$1,250}{\$20}\\
	&= 62.5\\
	\text{Broker will sell} &\text{ off 63 shares to get the margin back up to 40\%}
\end{align*}

\section{}
\subsection{Monthly Returns}
\subsubsection{End of March to End of April}
\begin{align*}
	\text{Adjusted (cum-)price} &= \frac{11}{10}\times\text{ex-price}\\
	&= \frac{11}{10}\times5 = \$5.5\\
	&\\
	\text{Return} &= \frac{5.5}{4.5}-1 = 22.2\%
\end{align*}

\subsubsection{End of April to End of May}
\begin{align*}
	\text{Adjusted (cum-)price} &= 4 + 0.1 = \$4.10\\
	&\\
	\text{Return} &= \frac{4.10}{5}-1 = -18\%
\end{align*}

\subsubsection{End of May to End of June}
\begin{align*}
	\text{Value of a right per share} &= \frac{P_{\text{ex-right}}-S}{N}\\
	&= \frac{4.6-3}{5} = \$0.32\\
	\text{Adjusted (cum-)price} &= 4.6 + 0.32 = \$4.92\\
	&\\
	\text{Return} &= \frac{4.92}{4}-1 = 23\%
\end{align*}

\subsection{Monthly geometric mean return (March-end till June-end)}
\begin{align*}
	r_G &= \big[(1+\text{HPR}_1)\times(1+\text{HPR}_2)\times...\times(1+\text{HPR}_n)\big]^{\frac{1}{n}} - 1\\
	&= \big[(1+0.2222)\times(1-0.18)\times(1+0.23)\big]^{\frac{1}{3}}-1\\
	&= 0.07223 = 7.22\%
\end{align*}

\section{}
\subsection{Computations}
\subsubsection{Standard Deviations of Company A and B}
\begin{align*}
	E\big[r_A\big] &= 0.05(-0.2) + 0.25(0) + 0.35(0.1) + 0.2(0.15) + 0.15(0.3) = 0.1 \text{ or } 10\%\\
	E\big[r_B\big] &= 0.05(-0.4) + 0.25(0.1) + 0.35(0) + 0.2(0.25) + 0.15(0.3) = 0.1 \text{ or } 10\%\\
	&\\
	Var\big[A\big] &= 0.05(-0.2-0.1)^2 + 0.25(0-0.1)^2 + 0.35(0.1-0.1)^2\\
	&\text{\qquad} + 0.2(0.15-0.1)^2 + 0.15(0.3-0.1)^2 = 0.0135 \text{ or } 135\%^2\\
	SD\big[A\big] &= \sqrt{Var\big[A\big]} = \sqrt{0.0135} = 0.1162 \text{ or } 11.62\%\\
	&\\
	Var\big[B\big] &= 0.05(-0.4-0.1)^2 + 0.25(0.1-0.1)^2 + 0.35(0-0.1)^2\\ 
	&\text{\qquad} + 0.2(0.25-0.1)^2 + 0.15(0.3-0.1)^2 = 0.0265 \text{ or } 265\%^2\\
	SD\big[B\big] &= \sqrt{Var\big[B\big]} = \sqrt{0.0265} = 0.1628 \text{ or } 16.28\%\\
	% E\big[r_A\big] &= 0.05() + 0.25() + 0.35() + 0.2() + 0.15()
\end{align*}

\subsubsection{Covariance of returns between Company A and B}
\begin{align*}
	\sigma_{AB} = \textit{Cov}(A,B) &= \sum^{m}_{i=1}\bigg(r_{Ai}-E\big[r_A\big]\bigg)\bigg(r_{Bi}-E\big[r_B\big]\bigg)p_i\\
	&=0.05(-0.2-0.1)(-0.4-0.1)\\
	&\text{\qquad}+0.25(0-0.1)(0.1-0.1)\\
	&\text{\qquad}+0.35(0.1-0.1)(0-0.1)\\
	&\text{\qquad}+0.2(0.15-0.1)(0.25-0.1)\\
	&\text{\qquad}+0.15(0.3-0.1)(0.3-0.1)\\
	&= 0.015 \text{ or } 150\%^2
\end{align*}

\subsubsection{Correlation between Company A and B}
\begin{align*}
	\rho_{AB} = \textit{Corr}(A,B) &= \frac{\sigma_{AB}}{\sigma_A\sigma_B}\\
	& = \frac{0.015}{0.1162\times0.1628} = 0.7931
\end{align*}

\subsection{}
The correlation coefficient bewteen the stocks of Company A and Company B is 0.793. This strongly positive correlation means that the returns of the two stocks will largely move in the same direction. Building a portfolio consisting only of A and B will provide some diversification benefit and mitigate risk (since they are not perfectly correlated), thus it is advised to invest in both stocks instead of solely investing in one.

However, since the correlation coefficient of A and B is already quite high, the most advisable option would be to build a portfolio with more stocks than just A and B.
\subsection{}
\begin{align*}
	Var\big[r_p\big] &= w_A^2Var(r_A) + w_B^2Var(r_B) + 2w_Aw_B\textit{Cov}(r_A,r_B)\\
	&\\
	\text{But note that } w_B = 1-w_A& \text{ Hence:}\\
	Var\big[r_p\big] &= w_A^2Var(r_A) + (1-w_A)^2Var(r_B) + 2w_A(1-w_A)\textit{Cov}(r_A,r_B)\\
	&= w_A^2(0.0135) + (1-w_A)^2(0.0265) + 2w_A(1-w_A)(0.015)\\
	&= 0.01w_A^2 - 0.023w_A + 0.0265\\
	&\\
	\text{Differentiate and solve for }&\text{$w_A$:}\\
	0.02w_A - 0.023 &= 0\\
	w_A &= 1.15\\
	w_B &= 1-1.0375 = -0.15 \\
\end{align*}
\begin{align*}
	\text{Alternatively find $w_A$ and }&\text{$w_B$ as such:}\\
	w_a &= \frac{\sigma_B^2-\rho_{AB}\sigma_A\sigma_B}{\sigma_A^2+\sigma_B^2-2\rho_{AB}\sigma_A\sigma_B}\\
	&= \frac{0.0265-(0.7931)(0.1162)(0.1628)}{0.135+0.0265-2(0.7931)(0.1162)(0.1628)}\\
	&= 1.15\\
	w_B &= 1 - w_A = -0.15\\
	&\\
	\text{Hence minimum variance }&\text{is:}\\
	Var\big[r_p\big] &= (1.15)^2(0.0135) + (-0.15)^2(0.0265) + 2(1.15)(-0.15)(0.015)\\
	&= 0.013275 \text{ or } 132.75\%^2
\end{align*}

\section{}
\subsection{A better portfolio}
The client's current portfolio sees him invested 100\% in Stock 1. His current portfolio's characteristics are defined below:
\begin{align*}
	E\big[R_p^{\text{(status quo)}}\big] &= 0.1 \text{ or } 10\%\\
	\sigma = SD\big[R_p^{\text{(status quo)}}\big] &= 0.1 \text{ or } 10\%\\
	\text{Hence: \qquad}\sigma^2 = Var\big[R_p^{\text{(status quo)}}\big] &= 0.01 \text{ or } 100\%^2\\
\end{align*}

A new, "better" portfolio is defined as one which has any of the following characteristics:
\begin{align*}
	E\big[R_p^{\text{(new)}}\big] &> E\big[R_p^{\text{(status quo)}}\big]\\
	\text{and:\quad}SD\big[R_p^{\text{(new)}}\big] &= SD\big[R_p^{\text{(status quo)}}\big]\\
	&\text{OR}\\
	E\big[R_p^{\text{(new)}}\big] &= E\big[R_p^{\text{(status quo)}}\big]\\
	\text{and:\quad}SD\big[R_p^{\text{(new)}}\big] &< SD\big[R_p^{\text{(status quo)}}\big]\\
	&\text{OR}\\
	E\big[R_p^{\text{(new)}}\big] &> E\big[R_p^{\text{(status quo)}}\big]\\
	\text{and:\quad}SD\big[R_p^{\text{(new)}}\big] &< SD\big[R_p^{\text{(status quo)}}\big]\\
\end{align*}

\newpage
First, we find the optimal risky portfolio weight for Stock 1 (and Stock 2) to form CAL when Stocks 1 and 2 are risky:
\begin{align*}
	w_1 &= \frac{\big(E\big[r_1\big]-r_f\big)\sigma_2^2-\big(E\big[r_2\big]-r_f\big)\sigma_1\sigma_2\rho_{1,2}}{\big(E\big[r_1\big]-r_f\big)\sigma_2^2 + \big(E\big[r_2\big]-r_f\big)\sigma_1^2-\big(E\big[r_1\big]-r_f+E\big[r_2\big]-r_f\big)\sigma_1\sigma_2\rho_{1,2}}\\
	&\\
	&= \frac{(0.1-0.05)\cdot0.3^2-(0.2-0.05)\cdot0.1\cdot0.3\cdot0.5}{(0.1-0.05)\cdot0.3^2+(0.2-0.05)\cdot0.1^2-(0.1-0.05+0.2-0.05)\cdot0.1\cdot0.3\cdot0.5}\\
	&= \frac{0.00225}{0.003} = 0.28125\\
	&\text{\qquad\qquad\qquad\qquad\qquad} w_1 = 0.75 \text{\qquad\qquad} w_2 = 0.25\\
\end{align*}
The characteristics for this optimal risky portfolio (ORP) are as follows:
\begin{align*}
	E\big[R_{ORP}\big] &= (0.75\times0.1) + (0.25*0.2) = 0.125 \text{ or } 12.5\%\\
	Var\big[R_{ORP}\big] &= (0.75^2\times0.1^2)+(0.25^2\times0.3^2) + (2\times0.75\times0.25\times0.5\times0.1\times0.3)\\
	&= 0.016875 \text{ or } 168.75\%^2\\
	SD\big[R_{ORP}\big] &= \sqrt{0.016875} = 0.1299 \text { or } 12.99\%\\
\end{align*}

Since this still doesn't meet our criteria for a "better" portfolio, we may choose to include the risk-free T-Bills by arbitrarily setting $w_f$ to 0.3. The new portfolio characteristics are as follows:
\begin{align*}
	&\text{\quad} w_f = 0.3 \text{\qquad\qquad} w_{ORP} = 0.7\\
	&\\
	E\big[R_{ORP}\big] &= (0.3\times0.05) + (0.7*0.125) = 0.1025 \text{ or }10.25\%\\
	Var\big[R_{ORP}\big] &= w_{ORP}^2Var\big[R_{ORP}\big]) = 0.00826875 \text{ or } 82.69\%^2\\
	SD\big[R_{ORP}\big] &= \sqrt{0.00826875} = 0.090932 \text { or } 9.09\%\\
\end{align*}

Overall, the weights are applied are as follows, leading to the new portfolio shown below:
\begin{align*}
	w_f &= 0.3 \text{\qquad\qquad} w_{ORP} = 0.7\\
	\text{Hence: \quad} w_1 &= 0.7\times0.75 = 0.525\\
	w_2 &= 0.7\times0.25 = 0.175\\
	&\\
	\text{Therefore, Amt in T-Bills} &= 0.3 \times \$10,000 = \$3,000\\
	\text{Amt in Stock 1} &= 0.525 \times \$10,000 = \$5,250\\
	\text{Amt in Stock 2} &= 0.175 \times \$10,000 = \$1,750\\
\end{align*}

The new portfolio shown above is better since it satisfies one of the criteria. It offers a higher expected return at 10.25\%, with a lower standard deviation of 9.1\%. This means the return is greater and the risk is lower for the customer than investing only in Stock 1.

This new portfolio is one of many feasible solutions to our criteria listed above. There are more sophisticated methods of calculating the optimal portfolio(s) given these three securities and their characteristics. That optimal portfolio(s) may or may not include all three securities. Further, the customer's risk appetite should also be factored in when designing a portfolio.

\subsection{Better Results Guaranteed?}
This portfolio does not guarantee anything. It simple offers statistically better prospects of generating a higher expected return while taking on a lower level of risk when compared with his original portfolio. 

On average this portfolio will perform comparatively better than his original portfolio (earning higher return and having a lower level of risk since this new portfolio diversifies away some of the idiosyncratic risk of holding only Stock 1). However, there is no guarantee of this actually happening---there is only an expectation that it will perform better. Actual returns may be higher or lower than the expected 10.25\%.

If, for instance, either Stock 2 or T-Bills drop significantly due to unforeseen circumstances, then this new portfolio's result will be worse than that of being invested in Stock 1 only.

My response: "No guarantees made on diversification. Simply better 'odds' of success (and lower chances of failure) than not being diversified."

\section{}

\end{document}
