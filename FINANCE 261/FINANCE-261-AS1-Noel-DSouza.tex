\documentclass[12pt]{article}

\usepackage[utf8]{inputenc} % Required for inputting international characters
\usepackage[T1]{fontenc} % Output font encoding for international characters
\usepackage{mathpazo} % Palatino font


%%% PAGE DIMENSIONS
\usepackage[margin=2.54cm]{geometry} % 1.27 is narrower!
\geometry{a4paper} 

%%% Font set to times new roman
\usepackage{times}
%%% Really good graphics package
\usepackage{graphicx}

% Insert {PDFS}
\usepackage{pdfpages}

% Change section spacing, remove the spacing
\usepackage[compact]{titlesec}
\titlespacing*{\section}{0pt}{*0}{*0}
\titlespacing*{\subsection}{0pt}{*0}{*0}
\titlespacing*{\subsubsection}{0pt}{*0}{*0}

%%% Better floats
\usepackage{float}


\usepackage{tikz}
\usepackage{wrapfig}

%%% PACKAGES
\usepackage{booktabs} % for much better looking tables
\usepackage{amsmath} % for better maths
\usepackage{paralist} % very flexible & customisable lists (eg. enumerate/itemize, etc.)
\usepackage{verbatim} % adds environment for commenting out blocks of text & for better verbatim
\usepackage{subfig} % make it possible to include more than one captioned figure/table in a single float
\usepackage{mathtools} % matrices
\usepackage{blkarray, bigstrut} % matrices

\usepackage[framed,numbered]{matlab-prettifier} % enable inserting matlab code.
\usepackage[parfill]{parskip}
%\addtolength{\jot}{1em}
\usepackage{amssymb}
\usepackage{cancel}
\usepackage{color}
\usepackage{listings}
\usepackage{pdflscape} % Enabling rotating pages into a landscape orientation.


%%% HEADERS & FOOTERS
\usepackage{fancyhdr} % This should be set AFTER setting up the page geometry
\pagestyle{fancy} % options: empty , plain , fancy
\renewcommand{\headrulewidth}{0pt} % customise the layout...
\lhead{}\chead{}\rhead{}
\lfoot{}\cfoot{\thepage}\rfoot{}

%%% SECTION TITLE APPEARANCE
\usepackage{sectsty}
\renewcommand{\thesubsection}{\thesection.\alph{subsection}}

%%% ToC (table of contents) APPEARANCE
\usepackage[nottoc,notlof,notlot]{tocbibind} % Put the bibliography in the ToC
\usepackage[titles,subfigure]{tocloft} % Alter the style of the Table of Contents
\renewcommand{\cftsecfont}{\rmfamily\mdseries\upshape}
\renewcommand{\cftsecpagefont}{\rmfamily\mdseries\upshape} % No bold!

%%% Hyperlinking 
\usepackage{hyperref}
\begin{document}

\begin{titlepage} % Suppresses displaying the page number on the title page and the subsequent page counts as page 1
	\newcommand{\HRule}{\rule{\linewidth}{0.5mm}} % Defines a new command for horizontal lines, change thickness here
	\center % Centre everything on the page
	
	%------------------------------------------------
	%	Headings
    %------------------------------------------------
    
	\textsc{\LARGE 2019}\\[1.5cm] % Main heading such as the name of your university/college
	\textsc{\Large Semester 1}\\[0.5cm] % Major heading such as course name
	% \textsc{\large Department of Engineering Science}\\[0.5cm] % Minor heading such as course title
	
	%------------------------------------------------
	%	Title
	%------------------------------------------------
	
	\HRule\\[0.4cm]
	{\huge\bfseries FINANCE 261 --- Assignment 1}\\[0.4cm] % Title of your document
	\HRule\\[1cm]
	
	%------------------------------------------------
	%	Author(s)
	%------------------------------------------------
	
    \begin{minipage}[t]{0.4\textwidth}
        \vspace{0pt}
		\begin{center}
			\large
			Noel \textsc{D'Souza}\\ % Your name
			\textit{ndso092}\\
			\textit{449609993}
		\end{center}
	\end{minipage}
	~
    % \begin{minipage}[t]{0.4\textwidth}
    %     \vspace{0pt}\raggedright
	% 	\begin{flushright}
	% 		\large
	% 		\textit{Employer Details}\\
    %         PwC \textsc{New Zealand} % Supervisor's name
    %         \linebreak \linebreak
    %         PwC Tower\\
    %         Level 22, 188 Quay St\\
    %         Auckland, 1010
	% 	\end{flushright}
	% \end{minipage}

	% If you don't want a supervisor, uncomment the two lines below and comment the code above
	%{\large\textit{Author}}\\
	%John \textsc{Smith} % Your name
	
	%------------------------------------------------
	%	Date
	%------------------------------------------------
	
	\vfill\vfill\vfill % Position the date 3/4 down the remaining page
	
    {\large\today} % Date, change the \today to a set date if you want to be precise
    % {\large 7 August, 2019}
	
	%------------------------------------------------
	%	Logo
	%------------------------------------------------
	
    % \vfill
    
    % \begin{minipage}[t]{0.4\textwidth}
    %     \vspace{0pt}
    %     \begin{flushleft}
    %     \includegraphics[width=0.9\textwidth]{engtitle.png}\\[1cm]
	% 	\end{flushleft}
	% \end{minipage}
	% ~
    % \begin{minipage}[t]{0.4\textwidth}
    %     \vspace{0pt}\raggedright
	% 	\begin{flushright}
    %         \includegraphics[width=0.45\textwidth]{pwc.png}\\[1cm]        
	% 	\end{flushright}
	% \end{minipage}


    % \includegraphics[width=0.4\textwidth]{engtitle.png}\\[1cm] % Include a department/university logo - this will require the graphicx package
	 
	%----------------------------------------------------------------------------------------
	
	\vfill % Push the date up 1/4 of the remaining page
	
\end{titlepage}

%------------------------------------------------
%   Summary
%------------------------------------------------

\section{}
\subsection{}
A 1\% increase in the price of \textbf{Stock C} will have the greatest impact on the value of the index. This is because a Price Weighted Index is comprised of one share of each stock. Hence a 1\% increase in the price of the stock with the \textbf{largest share price} (i.e. Stock C) will contribute the most towards the index value.

\subsection{Divisor at t=0}
\begin{align*}
	\frac{P_0(A) + P_0(B) + P_0(C)}{D} &= I_0\\
	\frac{40+50+100}{D} &= 100\\
	D &= \frac{190}{100} \\
	D &= 1.9\\
\end{align*}

\subsection{Price weighted index value at t=1, and monthly \newline percentage return}
\begin{align*}
	\frac{P_1(A) + P_1(B) + P_1(C)}{D} &= I_1\\
	\frac{44+54+100}{1.9} &= I_1\\
	I_1 &= \frac{208}{1.9} \\
	I_1 &= 109.47 \text{ index value after 1 month}\\
	&\\
	\text{Monthly Return} &= \frac{109.47-100}{100}\\
	&= 9.47\%
\end{align*}

\subsection{Divisior after 2-for-1 stock split for stock C}
\begin{align*}
	\frac{P_1(A) + P_1(B) + P_1^{\text{(\textit{split})}}(C)}{D^{(\textit{split})}} &= I_1\\
	\frac{44+54+55}{D^{(\textit{split})}} &= 109.47\\
	D^{(\textit{split})} &= \frac{153}{109.47} \\
	D^{(\textit{split})} &= 1.40\\
\end{align*}

\section{}
\subsection{Initial investment and borrowing}
\begin{align*}
	\text{Position Value} &= 1000 \text{ shares} \times \$25 \text{ per share}\\
	&= \$25,000\\
	&\\
	\text{Initial Investor Equity} &= \text{Position Value} \times \text{IMR}\\
	&= \$25,000 \times 0.5\\
	&= \$12,500\\
	&\\
	\text{Amount Borrowed} &= \text{Position Value} \times (1-\text{IMR})\\
	&= \$25,000 \times (1-0.5)\\
	&= \$12,500\\
\end{align*}

\subsection{Margin call trigger price}
\begin{align*}
	\text{Maintenance Margin} &=  \frac{\text{Total Position Value} - \text{Borrowing}}{\text{Total Position Value}} \\
	0.4 &= \frac{1,000p - 12,500}{1,000p}\\
	400p &= 1,000p-12,500\\
	p &= \frac{12,500}{600}\\
	p &= \$20.83 \text{ per share is the margin call trigger price} 
\end{align*}

\subsection{Pledging T-bills}
\begin{align*}
	\text{Maintenance Margin} &=  \frac{\text{Total Value} - \text{Borrowing} + \text{Additional Contribution of T-bills}}{\text{Total Value}+ \text{Additional Contribution of T-bills}} \\
	0.4 &= \frac{(1,000\times20) - 12,500 + x}{(1,000\times20)+x} \\
	8,000 + 0.4x &= 7,500 + x\\
	0.6x &= 500\\
	x &= \$833.33 \text{ worth of T-bills must be added}\\
\end{align*}

\subsection{Extent of partial liquidation}
\begin{align*}
	\text{Maintenance Margin} &=  \frac{\text{Total Value} - \text{Borrowing}}{\text{Total Value} - \text{Value of Liquidated Shares}} \\
	0.4 &= \frac{(1,000\times20) - 12,500}{(1,000\times20)-x} \\
	8,000 - 0.4x &= 7,500\\
	0.4x &= 500\\
	x &= \$1,250 \text{ is the value of shares sold off}\\
\end{align*}

\begin{flushleft}
	\text{Given the share price of \$20 per share:}
\end{flushleft}

\begin{align*}
	\text{Number of shares sold off by broker} &= \frac{\$1,250}{\$20}\\
	&= 62.5\\
	\text{Broker will sell} &\text{ off 63 shares to get the margin back up to 40\%}
\end{align*}

\section{}
\subsection{}
\section{}
\section{}
\section{}



\end{document}
